\section{Risoluzione delle interrogazioni}
La risoluzione di un'interrogazione dipende fondamentalmente da due fattori:
\begin{itemize}
    \item Il linguaggio di interrogazione e la corrispondente semantica.
    \item Il tipo di dati contenuti nelle affissioni dell'indice.
\end{itemize}
Per il momento, ci concentreremo sulla risoluzione di interrogazioni \textit{booleane}, cioè in cui gli unici operatori sono la congiunzione, la disgiunzione e la negazione logica. Una formula del linguaggio di interrogazione è: un termine oppure la congiunzione, disgiunzione o negazione di formule.\\
Per quanto riguarda la semantica:
\begin{itemize}
    \item La semantica di una formula è una lista di documenti.
    \item La semantica di un termine è data dalla lista di documenti in cui il termine compare.
    \item La semantica della congiunzione è data dall'intersezione delle liste.
    \item La semantica della disgiunzione è data dalla fusione delle liste.
    \item La semantica della negazione è data dalla complementazione rispetto all'intera collezione documentale.
\end{itemize}
Chiaramente, per risolvere un'interrogazione Booleana è sufficiente che le affissioni contengano i puntatori documentali. Inoltre, se l'indice contiene i puntatori in ordine crescente, è possibile ottenere intersezione e unione di liste in tempo lineare nel numero di puntatori.\\
È di grande interesse operare in maniera \textit{pigra} (document-at-a-time) - leggere cioè dalle liste in input solo i puntatori necessari a emettere un certo prefisso dell'output. Questo permette di gestire in maniera efficiente la \textit{early termination}, tecnica con cui il calcolo della semantica viene arrestato sulla base di stime della qualità dei documenti già rinvenuti.\\
Per fondere liste in tempo lineare (con perdita logaritmica nel numero di liste in ingresso) è sufficiente una \textit{coda di priorità indiretta} che contiene puntatori alle liste, prioritizzate secondo il documento corrente. A ogni passo, il prossimo documento restituito sarà il documento corrente della lista in cima alla coda (cioè quella che correntemente è posizionata sul documento più piccolo). Una volta deciso il documento, avanziamo la lista in cima alla coda e aggiustiamo la coda stessa finchè la lista in cima non è posizionata su un documento diverso. Si noti che in questo modo siamo anche in grado di sapere \textit{quali} liste contengono il documento corrente.\\
Per quanto riguarda l'intersezione, sebbene in linea di principio il limite inferiore lineare non sia superabile, esistono diverse euristiche che permettono di accelerare la computazione nel caso l'indice fornisca la possibilità di effettuare \textit{salti}.\\
In linea di principio, potremmo sempre tenere conto tramite una coda di priorità indiretta del documento minimo corrente, come nel caso precedente, ma tenere al tempo stesso traccia del massimo puntatore in una variabile\footnote{Supponendo che la coda a priorità indiretta sia implementata tramite uno Heap (per esempio) recuperare il massimo puntatore documentale richiede tempo e spazio costante perchè il valore più a destra nella struttura è sempre il massimo corrente.}. Quando coincidono, il puntatore sta nell'intersezione e può essere restituito (e immediatamente dopo, una qualunque lista viene fatta avanzare). Altrimenti la lista che realizza il minimo viene fatta saltare fino al massimo e la coda viene aggiornata. Dato che viene effettuato al più un aggiornamento per avanzamento, il tempo richiesto è lineare nella dimensione di input, con una perdita logaritmica nel numero di input.\\
Da un punto di vista pratico, però, è spesso più efficiente utilizzare una soluzione che ha in linea teorica un comportamento molto peggiore (la perdita nel numero di input è \textit{lineare}). L'idea è di far avanzare in maniera miope le liste in ordine fino al massimo corrente $m$ (all'inizio, il primo puntatore della prima lista) finchè non sono tutte allineate. Al primo disallineamento (che causa un incremento di $m$) si ricomincia ad allineare la prima lista. Quando si arriva ad allineare l'ultima lista, si ha un puntatore da restituire.\\
L'aspetto interessante di questo approccio è che ordinando le liste in ordine di frequenza crescente (quando è nota - per esempio nel caso di termini) la maggior parte degli avanzamenti verrà giocato dalle prime liste, che essendo quelle di minima frequenza genereranno pochi allineamenti. Le liste più dense subiranno pochi avanzamenti molto consistenti, che potranno essere gestiti in maniera efficienti tramite un sistema di salti. È anche possibile implementare una versione adattiva che mano a mano che scandisce le liste in input (non necessariamente liste di termini) ne stima la frequenza e le riordina al volo.\\
Si noti che nel caso le liste siano interamente caricabili in memoria, il problema diventa completamente diverso, ed esistono soluzioni molto sofisticate per il problema dell'intersezione \cite{intersection}.\\
L'osservazione fondamentale è che intersecare due liste, se una è molto più corta dell'altra (per esempio esponenzialmente più corta), può essere conveniente cercare con una ricerca dicotomica gli elementi della lista più corta in quella più lunga.\\
Questa osservazione permette di risolvere rapidamente in memoria centrale, in maniera non pigra (term-at-a-time), intersezioni di liste. È sufficiente ordinare le liste in ordine di frequenza (questo è sempre possibile, essendo l'algoritmo non pigro) e procedere a ridurre, una lista alla volta, ma la speranza è che quando arriveremo alle liste più lunghe (le ultime) i candidati siano così pochi che sia possibile utilizzare tecniche di intersezione come quelle summenzionate.
\subsection{Indici distribuiti}
Le dimensioni degli indici del web sono tali da rendere poco pratica la memorizzazione dell'intero indice in una sola macchina. È quindi necessario \textit{segmentare} l'indice complessivo in sottoindici, detti \textit{segmenti}, le cui risposte verranno poi opportunamente combinate. È sottointeso che i segmenti saranno in genere memorizzati su un gruppo di macchine collegate in rete, e che un opportuno protocollo di comunicazione permetterà di accedere al contenuto dei segmenti in una macchina remota.\\
In linea di principio, un indice può essere partizionato tramite una funzione che assegna a ogni affissione un segmento destinazione. A ogni segmento saranno associati i termini per i quali compare almeno un'affissione.\\
In pratica però è più comune scegliere un singolo criterio di partizionamento - \textit{lessicale} o \textit{documentale} - su cui basare il processo. In un partizionamento documentale, la collezione documentale viene divisa in blocchi (non necessariamente contigui) e a ogni blocco viene assegnato un segmento. Tutte le affissioni relative a un blocco andranno a finire nel suo segmento, e al segmento saranno associati i termini che compaiono nel blocco. Di norma, lo stesso termine comparirà in più segmenti (si pensi a termini comuni come le congiunzioni), mentre gli \textit{hapax legomena} compariranno esattamente in un solo segmento.\\
In alternativa, è possibile partizionare la collezione documentale lessicalmente: viene scelto un criterio per dividere l'insieme dei termini in più blocchi (non necessariamente lessicograficamente contigui), e a ogni blocco viene associato un segmento: la lista di affissioni relativa a un termine sarà interamente contenuta nel segmento relativo.\\
La ricostruzione delle liste di affissione a partire da una segmentazione lessicale è banale: basta individuare il segmento che contiene il termine e interrogarlo. L'individuazione del segmento può non essere un problema banale se, ad esempio, i blocchi di termini non sono lessicograficamente consecutivi. Nel caso peggiore, l'unica soluzione è l'interrogazione di tutti i segmenti.\\
La ricostruzione delle liste di affissione a partire da una segmentazione documentale è più delicata, dato che nella maggior parte dei casi sarà necessario leggere vari frammenti della lista complessiva e combinarli. Questo richiede \textit{in primis} di interrogare tutti i segmenti per sapere quali possiedono una lista di affissione per il termine dato. Le liste vanno poi combinate - con una semplice concatenazione e rinumerazione se i blocchi di documenti sono consecutivi, o con una fusione di liste in caso contrario.\\
In entrambi i casi precedenti può essere utile filtrare le richieste ai segmenti tramite dei dizionari approssimati, che possono rappresentare in maniera molto compatta l'insieme presente in ogni segmento. Dimensionando i dizionari a seconda della memoria disponibile è possibile ridurre il numero di richieste inutili fatte ai segmenti.\\
Va notata una caratteristica importante del partizionamento documentale: è possibile cioè risolvere \textit{direttamente} un'interrogazione complessa a livello di segmento. Questo fatto può portare a un miglioramento netto delle prestazioni, perchè ad esempio, in presenza di un'interrogazione congiunta di termini, alcuni segmenti, pur contenendo tutti i termini dell'interrogazione, potrebbero non restituire nessun documento. Nel caso di un'interrogazione disgiunta, non c'è in ogni caso riduzione delle prestazioni. Va però fatto notare che se il processo di risoluzione della query restituisce artefatti (quali valori di ranking, calcolo della prossimità, etc\dots) il protocollo di rete con cui ci si connette all'indice dovrà essere in grado di trasmetterli. Detto altrimenti, risolvendo e trasmettendo solo liste di affissioni, il protocollo dipenderà solo dalla struttura dell'indice, mentre la risposta a interrogazioni con punteggio lo renderà dipendente dal meccanismo di assegnamento del punteggio.\\
D'altra parte, il partizionamento lessicale offre un'interessante possibilità: quella cioè di tenere in memoria centrale la parte dell'indice corrispondente ai termini più interessanti in qualche senso, o che risultano più utilizzati da una rilevazione empirica.