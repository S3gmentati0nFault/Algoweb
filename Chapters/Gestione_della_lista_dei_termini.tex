\section{Gestione della lista dei termini}
La lista dei termini di un indice può essere molto ingombrante. Esistono diversi metodi per rappresentarla: in questo contesto, ne vedremo uno piuttosto sofisticato, che ha il grosso vantaggio di essere in grado, data una lista di stringhe, di restituire il numero ordinale di qualunque elemento della lista, ma di \textit{non memorizzare la lista stessa}: in effetti, l'occupazione di memoria sarà di $1,23n$ interi di $\log(n)$ bit, dove $n$ è il numero delle stringhe.\\
Una funzione di hash per un insieme di chiavi \set{X} $\subseteq$ \set{U}, dove \set{U} è l'universo di tutte le chiavi possibili, è una funzione \fun{f}{X}{m} detta:
\begin{itemize}
    \item \textit{perfetta} se la funzione è iniettiva\footnote{Ricordo: una funzione è iniettiva se per ogni elemento nel codominio esiste al più un elemento nel dominio, tale che $x R y$. In questo caso significa che la funzione di hash non presenta collisioni.}.
    \item \textit{minimale} se $|X| = m$.
    \item \textit{preserva l'ordine} se esiste un ordine lineare su \set{X} e si ha $x \leq y \iff f(x) \leq f(y)$.  
\end{itemize}
Il nostro scopo è quello di costruire, per un insieme di stringhe arbitrario, una funzione di hash minimale, perfetta, e che preservi l'ordine. (si noti che di per sè non stiamo chiedendo qual è il risultato di $f$ su $U \smallsetminus X$: questo problema verrà risolto a parte).\\
Esistono numerose tecniche per la costruzione di hash perfetti. % INSERIRE LA CITAZIONE %
Quello che andiamo a descrivere è basato sulla teoria dei grafi casuali, e utilizza la randomizzazione per la costruzione della struttura dati. Quella che descriveremo è in effetti una tecnica completamente generale per memorizzare una funzione statica (immutabile) \fun{f}{X}{2^b} dall'insieme \set{X} all'insieme dei valori $2^b$. Scegliendo $b = \ceil{\log(n)}$ e mappando ogni termine in un indice al suo rango lessicografico, otterremo il risultato richiesto.\\
L'idea della costruzione è la seguente: prendiamo due funzioni di hash aleatorie \fun{h}{U}{m} e \fun{g}{U}{m}, dove $m \geq n$ e $n$ è la cardinalità di \set{X}, e consideriamo un vettore \textbf{w} di interi a $b$ bit che sia soluzione del sistema
\begin{equation*}
    w_{h(x)} \oplus w_{g(x)} = f(x) \hspace{5mm} x \in X
\end{equation*}
Chiaramente, memorizzando $h$, $g$ e il vettore \textbf{w} abbiamo memorizzato $f$: per calcolare $f(x)$ basta semplicemente risolvere il sistema, che è di $n$ equazioni con $w_i$ incognite, che, però, potrebbe non avere soluzione. Il sistema deve essere risolto una sola volta, in fase di inizializzazione della struttura, e poi potremo accedere a tutti i valori associati alla funzione $f(x)$ in tempo lineare nel numero di funzioni di hash (calcolo le funzioni di hash, accedo alle posizioni del vettore e calcolo lo XOR dei valori). Possiamo rappresentare i vincoli imposti dal sistema come segue: costruiamo un grafo G non orientato con $m$ vertici e $n$ lati in cui, per ogni $x \in X$, abbiamo che $h(x)$ è adiacente a $g(x)$ tramite un lato etichettato da $x$.\\
Andiamo ora a eseguire l'operazione di \textit{esfoliazione} dell grafo così costruito: partendo da una qualunque foglia $v$ (un vertice di grado 0 o 1), la rimuoviamo dal grafo. Se questa è adiacente a un altro vertice, rimuoviamo anche il lato $l$ che li connette, a questo punto mettiamo la coppia $\langle v, l \rangle$ dentro uno stack.\\
Il processo continua fino a quando il grafo non è vuoto o non si incontra un ciclo. Se non incontriamo alcun ciclo, significa che il sistema può essere risolto, questo perchè ogni equazione del sistema deve comparire una sola volta nello stack, altrimenti il processo si conclude quando rimuoviamo gli ultimi nodi, che avranno degree zero, e di conseguenza il loro valore $w_i$ sará pari a 0. Se risolviamo l'equazione associata a ogni elemento $\langle v, l \rangle$ sfruttando il fatto che $v = h(x) \lor v = g(x)$, e $v$ non è stato certamente assegnato precedentemente, dato che ogni vertice in una coppia non compare mai in coppie successive. Essenzialmente stiamo \textit{triangolando} la matrice associata al sistema (si vedano i lucidi animati online).\\
Ora, assumendo che $h$ e $g$ siano casuali e indipendenti, il grafo che andiamo a costruire è un grafo casuale di $m$ vertici con $n$ archi, e un risultato importante di teoria dei grafi casuali dice che per $n$ sufficientemente grande, quando $m > 2.09n$ il grafo è quasi sempre privo di cicli (questo significa che il rapporto tra il numero di grafi privi cicli e quelli totali tende a 1). In sostanza, scegliendo bene $h$ e $g$ quasi tutti i grafi che otterremo permetteranno di risolvere il sistema.\\
Il caso ora descritto non è in realtà ottimo. La teoria degli ipergrafi casuali ci dice che utilizzando 3-ipergrafi (cioè un insieme di sottoinsiemi di ordine 3 dell'insieme dei vertici) è possibile dare una nozione di aciclicità che permette di risolvere i sistemi nel caso di \textit{tre} funzioni di hash, ma in questo caso il limite che garantisce l'aciclicità è $m > 1.23n$ - un miglioramento netto rispetto al caso di ordine 2.
Alla fine, per memorizzare la funzione statica \fun{f}{X}{2^b} dovremo utilizzare solo 1.23 bit per elemento.
\subsection{Firme}
La funzione così costruita, per quanto sia minimale, perfetta e preservi l'ordine non permette di riconoscere se un elemento fa parte di \set{X} o no. Per ovviare all'inconveniente utilizziamo un insieme di \textit{firme} associato all'insieme \set{X} delle chiavi. Consideriamo cioè una funzione \fun{s}{U}{2^r} che associ a ogni chiave possibile una sequenza di $r$ bit "casuale", nel senso che la probabilità che $s(x) = s(y)$ se $x$ e $y$ sono presi uniformemente a caso da \set{U} è \set{2^{-r}} (per esempio una buona funzione di hash). Consideriamo l'enumerazione ordinata $x_0 \leq x_1 \leq \dots \leq x_{n - 1}$ degli elementi di \set{X}. Oltre a $f$, memorizziamo 
un vettore di dimensione $n$ in cui i valori di \set{X} sono mappati nelle posizioni $f(x)$, che é il rango lessicografico di $x$, affinché, preso $x \in X$, calcolati $f(x)$ ed $s(x)$ avró che
\begin{equation*}
    S_{f(x)} = s(x)
\end{equation*}
mentre se $x \notin X$ il valore di $f(x)$ sará arbitrario, quindi $S_{f(x)} = -1$ e il valore di $s(x)$ sará a sua volta arbitrario.\\
Per interrogare la struttura risultante su input $x \in U$, agiamo come segue:
\begin{enumerate}
    \item Calcoliamo $f(x)$ (che è un numero in $n$).
    \item Recuperiamo $S_{f(x)}$.
    \item Se $S_{f(x)} = s(x)$, restituiamo $f(x)$; altrimenti restituiamo -1 (a indicare che $x$ non fa parte di \set{X}).
\end{enumerate}
Si noti che se $x \in X$, $f(x)$ è il suo rango in \set{X}, e quindi $S_{f(x)}$ conterrà, per definizione, $s(x)$.
Se invece $x \notin X$, $S_{f(x)}$ sarà una firma arbitraria, e quindi restituiremo quasi sempre -1, tranne nel caso ci sia una collisione tra firme, il che avviene, come detto, con probabilità $2^{-r}$. Tarando il numero $r$ di bit delle firme, è quindi possibile bilanciare spazio occupato e precisione della struttura.\\
\subsection{Ottimizzazioni}
È possibile migliorare ulteriormente l'occupazione in spazio, quando $b$ non è troppo piccolo, utilizzando un \textit{vettore compatto} per memorizzare i valori della soluzione del sistema. In effetti, per come abbiamo descritto il processo di risoluzione, non è possibile che vengano poste a valori diversi da 0 più di $n$ variabili. Se potessimo memorizzare, con una piccola perdita di spazio, \textit{solo i valori diversi da zero} potremmo ridurre l'occupazione di memoria a quasi $nb$ bit.\\
Per farlo consideriamo un vettore di bit \textbf{b} che contiene in posizione $i$ un uno se la variabile corrispondente $w_i$ è diversa da zero, e zero altrimenti. Definiamo l'operazione \textit{rank}
\begin{equation*}
    rank(\textbf{b}, i) = |\{j < i \st b_j \neq 0\}|
\end{equation*}
che conta il numero di uno che compaiono a sinistra della posizione corrente. Chiaramente se mettiamo in un vettore \textbf{c} i valori delle variabili $w_i$ diversi da zero avremo che:
\begin{equation*}
    w_i =
    \begin{cases}
        0 \hspace{7mm} &\textnormal{se } b_i = 0\\
        \textbf{c}[rank(\textbf{b}, i)] \hspace{7mm} &\textnormal{se } b_i \neq 0\\
    \end{cases}
\end{equation*}
Quanto spazio occorre per calcolare rapidamente il rango? Esistono soluzioni molto sofisticate che occupano spazio $o(n)$\footnote{Un esempio non banale di strutture efficienti per la memorizzazione del rango è la struttura di Jacobson.}, ma qui proporremo un metodo molto semplice: scelto un quanto $q$, memorizziamo in una tabella $R$ i valori $rank(\textbf{b}, kq)$. Per calcolare il rango in posizione $p$, basta recuperare $R[\flr{p / q}]$ e completare il calcolo contando gli uno dalla posizione $q\flr{p / q}$ alla posizione $p$ (utilizzando le istruzioni di conteggio efficiente per le CPU moderne). Per $q$ fissato, il tempo del calcolo è costante. Lo spazio occupato dipende anch'esso da $q$, che permette quindi di bilanciare lo spazio occupato e la velocità.\\
Alla fine, la funzione occuperà $nb + 1.23n$ bit, più il necessario per la struttura di rango (supponendo $q = 512$ saranno necessari altri $0.125n$ bit). Si noti che è impossibile scrivere una funzione in meno di $n^2$ bit, dato che esistono $(2^b)^n$ funzioni da un insieme di $n$ a un insieme di $2^b$ elementi.
\subsection{Auto-firma}
Un utilizzo interessante delle funzioni che abbiamo descritto è quello di memorizzare efficientemente un dizionario statico approssimato. Per farlo, consideriamo un insieme $X \subseteq U$ che vogliamo rappresentare, e una funzione di firma \fun{s}{U}{2^r}.
Possiamo utilizzare la tecnica generale di rappresentazione delle funzioni per scrivere in $nr + 1.23n$ bit la funzione \fun{f}{X}{2^r} data da $f(x) = s(x)$. La funzione $f$, cioè, mappa ogni chiave nella propria firma. Il dizionario viene a questo punto interrogato come segue:
\begin{enumerate}
    \item Calcoliamo $f(x)$
    \item Se $f(x) = s(x)$, restituiamo "appartiene a \set{X}"; altrimenti, restituiamo "non appartiene a \set{X}".
\end{enumerate}
Si noti che, se $x \in X$, $f(x)$ è la sua firma $s(x)$, e quindi restituiremo "appartiene a \set{X}". Se invece $x \notin X$, $f(x)$ sarà un valore arbitrario, e quindi restituiremo "non appartiene a \set{X}", tranne nel caso in cui ci sia una collisione tra firme, il che avviene con probabilità $2^{-r}$. Tarando il numero $r$ di bit delle firme è quindi possibile bilanciare spazio occupato e precisione della struttura.\\
Rispetto a un filtro di Bloom, un dizionario di questo tipo è molto più compatto (un filtro di Bloom, per avere una precisione di $2^{-r}$ richiede 1.44 bit per elemento) e molto più veloce (le funzioni sono valutabili con al più tre fallimenti di cache, contro gli $r$ del filtro di Bloom). Per contro, non è modificabile.