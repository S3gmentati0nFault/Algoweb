\section{Codici istantanei}
Un \textit{codice} é un insieme $C \subseteq 2^*$, cioé un insieme di parole binarie. Si noti che per ovvie ragioni di cardinalitá \set{C} é al piú numerabile.\\
Definiamo l'\textit{ordinamento per prefissi} delle sequenze in \set{2^*} come segue:
\begin{equation}
    x \preceq y \iff \exists z \st y = xz
\end{equation}
Cioé $x \preceq y$ se e solo se $x$ é un prefisso di $y$. Ricordiamo che in un \textit{ordine parziale} due elementi sono inconfrontabili se nessuno dei due é minore dell'altro.\\
Un codice é detto \textit{istantaneo} o \textit{privo di prefissi} se ogni coppia di parole distinte del codice é inconfrontabile. L'effetto pratico di questa proprietá é che a fronte di una parola $w$ formata da una concatenazione di parole del codice, non esiste una diversa concatenazione che dá $w$. In particolare, leggendo uno a uno i bit di $w$ é possibile ottenere in maniera istantanea le parole del codice che lo compongono.\\
Ad esempio, il codice \set{\{0, 1\}} é istantaneo, mentre \set{\{0, 1, 01\}} non lo é. Se prendiamo ad esempio la stringa 001 e la confrontiamo con il primo codice sappiamo che é formata da 0, 0, 1, ma nel secondo caso possiamo scegliere tra 0, 0, 1 e 0, 01.\\
Un codice si dice \textit{completo} o \textit{non ridondante} se ogni parola $w \in 2^*$ é confrontabile con qualche parola del codice (esiste quindi una parola del codice di cui $w$ é prefisso o una parola del codice che é un prefisso di $w$). Il primo dei due codici summenzionato é completo, mentre il secondo non lo é.\\
Quando un codice istantaneo é completo, non é possibile aggiungere parole al codice senza perdere la proprietá di istantaneitá; inoltre, qualunque parola \textit{infinita} é decomponibile in maniera unica come una sequenza di parole del codice, qualunque parola \textit{finita} é decomponibile in maniera unica come sequenza di parole del codice piú un prefisso di qualche parola del codice.
\begin{teo}
    Sia $C \subseteq 2^*$ un codice, se \set{C} é istantaneo, allora
    \begin{equation*}
        \sum_{w \in C}{2^{-|w|}} \leq 1
    \end{equation*}
    \set{C} é completo se e solo se l'uguaglianza vale. Inoltre, data una sequenza, eventualmente infinita, $t_0, \dots, t_{n - 1}, \dots$ che soddisfa:
    \begin{equation*}
        \sum_{i \in n}{2^{-t_i}} \leq 1
    \end{equation*}
    esiste un codice istantaneo formato da parole $w_0, \dots, w_{n - 1}, \dots$ tali che $|w_i| = t_i$
\end{teo}
Prima di cominciare la dimostrazione facciamo un preambolo utile a eseguirla in modo veloce.\\
Un \textit{diadico} é un razionale della forma $kw^{-h}$. A ogni parola $w \in 2^*$. Possiamo associare un sottointervallo semiamperto di $[0\dots1)$ con estremi diadici come segue:
\begin{itemize}
    \item Se $w$ é la parola vuota, l'intervallo é $[0\dots1)$
    \item Se $w$ privata dell'ultimo bit ha $[x \dots y)$ come intervallo associato, se l'ultimo bit é 0 allora l'intervallo di $w$ é $[x \dots (x + y)/2)$, altrimenti l'intervallo di $w$ é $[(x + y)/2 \dots y)$
\end{itemize}
Per poter visualizzare quanto segue é utile costruire qualche esempio semplice e inserire le parole del codice in un albero, in cui ogni nodo ha al piú due figli etichettati 0 e 1. Si possono fare le seguenti \textbf{osservazioni}:
\begin{enumerate}
    \item L'intervallo associato a una parola di lunghezza $n$ ha lunghezza $2^{-n}$
    \item se $v \preceq w$ l'intervallo associato a $v$ contiene quello associato a $w$
    \item Due parole sono inconfrontabili se solo i corrispondenti intervalli sono disgiunti; infatti, se $v$ e $w$ sono inconfrontabili e $z$ é il loro massimo prefisso comune, assumendo senza perdita di generalitá che $z_0 \preceq v$ e $z_1 \preceq w$ l'intervallo di $v$ sará contenuto in quello di $z_0$ e l'intervallo di $w$ in quello di $z_1$: dato che gli intervalli di $z_0$ e $z_1$ sono disgiunti per definizione, lo sono anche quelli di $v$ e $w$
    \item Dato un qualunque intervallo diadico $[k2^{-h} \dots (k + 1)2^{-h})$ esiste un'unica parola di lunghezza $h$ a cui é associato l'intervallo, vale a dire, la parola formata dalla rappresentazione binaria di $k$ allineata ad $h$ bit; questo é certamente vero per $h = 0$, e data una parola $w$ con $|w| = h + 1$ se l'intervallo associato a $w$ privato dell'ultimo carattere é $[k2^{-h} \dots (k + 1)2^{-h})$ l'intervallo associato a $w$ é $[(2k)2^{-h-1} \dots (2k + 1)2^{-h-1})$; se l'ultimo carattere di $w$ é 0, $[(2k + 1)2^{-h-1} \dots 2(k + 1)2^{-h-1})$
\end{enumerate}
Fatte le dovute premesse possiamo passare alla dimostrazione del Teorema 1.\\
\begin{proof}
    Sia ora \set{C} un codice istantaneo. La sommatoria contenuta nell'enunciato del Teorema 1 é la somma delle lunghezze degli intervalli associati alle parole di \set{C}; questi intervalli sono disgiunti e la loro unione forma un sottoinsieme di $[0\dots1)$ che ha necessariamente lunghezza minore o uguale di 1.\\
    ($\Longrightarrow$) Se la sommatoria é strettamente minore di 1 deve esserci per forza un intervallo scoperto, diciamo $[x \dots y)$. Questo intervallo contiene necessariamente un sottointervallo della forma $[k2^{-h} \dots (k + 1)2^{-h})$ per qualche $h, k$. Ma allora la parola associata a quest'ultimo potrebbe essere aggiunta al codice (essendo inconfrontabile con le altre [osservazione 3]). Che quindi risulta essere incompleto.\\
    ($\Longleftarrow$) D'altra parte, se il codice é incompleto l'intervallo corrispondente a una parola inconfrontabile con tutte quelle del codice é necessariamente scoperto, e rende la somma strettamente minore di 1.\\
    Andiamo ora a dimostrare l'ultima parte dell'enunciato, assumendo, senza perdita di generalitá, che la sequenza $t_0, \dots, t_{n - 1}, \dots$ sia monotona non decrescente.\\
    Genereremo le parole  $w_0, \dots, w_{n - 1}, \dots$ in maniera \textit{miope}. Sia $d$ l'estremo sinistro della parte di intervallo unitario correntemente coperta dagli intervalli associati dalle parole giá generate: inizialmente, $d = 0$. Manterremo vero l'invariante che prima dell'emissione della parola $w_n$ si ha $d = k2^{-t_n}$ per qualche $k$, il che ci permetterá di scegliere come $w_n$ l'unica parola di lunghezza $t_n$ il cui intervallo ha estremo sinistro $d$. Dato che l'intervallo associato a ogni nuova parola é disgiunto dall'unione dei precedenti, le parole generate saranno tutte inconfrontabili.\\
    L'invariane é ovviamente vero quando $n = 0$. Dopo aver generato $w_n$, $d$ viene aggiornato sommandogli $2^{-t_n}$ e diventa quindi $(k + 1)2^{-t_n}$. Ma dato che $(k + 1)2^{-t_n} = ((k + 1)2^{t_{n + 1}-t_n})2^{-t_{n + 1}}$ e $t_{n + 1} \geq t_n$, l'invariante viene mantenuto.
    \qedhere
\end{proof}
\subsection{Codici istantanei per gli interi}
Alcuni dei metodi piú utilizzati per la compressione degli indici utilizzano codici istantanei per gli interi. Questa scelta puó apparire a prima vista opinabile per il fatto che i valori che compaiono in un indice hanno delle limitazioni superiori naturali e sono facili da caloclare, e quindi potrebbe essere piú efficiente calcolare un codice istantaneo per il solo sottoinsieme di interi effettivamente utilizzato.\\
In realtá se si lavora con collezioni documentali di grandi dimensioni la smplicitá teorica e implementativa dei codici per gli interi li rende molto interessanti.\\
Innanzitutto si noti che un codice istantaneo per gli interi é numerabile. L'associazione tra interi e parole del codice va specificata di volta in volta, anche se, in tutti i codici che vedremo, l'associazione é semplicemente data dall'ordinamento prima per lunchezza e poi lessicografico delle parole. Inoltre assumeremo che le parole rappresentino numeri naturali, e quindi la parola minima (cioé lessicograficamente minima tra quelle di lunghezza minima) rappresenti lo zero\footnote{Questa scelta non é uniforme in letteratura, e in effetti si possono trovare nello stesso libro due codici per gli interi che, a seconda della bisogna, vengono numerati a partire da zero o da uno}.
La rappresentazione piú elementare di un intero $n$ é quella \textit{binaria}, che peró non é istantanea (le prime parole sono 0, 1, 10, 11, 100). É possibile rendere il codice istantaneo facendo un allineamento a $k$ bit. La lunghezza di una parola di codice binario (non allineato) é $\lambda(n) + 1$\footnote{Ricordo che: $\lambda(n) = \flr{\log(n)}$}.\\
Chiameremo \textit{rappresentazione binaria ridotta} di $n$ la rappresentazione binaria di $n + 1$ alla quale viene rimosso il bit piú significativo; anch'essa non é istantanea. La lunghezza della parola di codice per $n$ é $\lambda(n + 1)$. Le prime parole sono $\varepsilon$, 0, 1, 00, 01, 10.\\
Un ruolo importante nella costruzione dei codici istantanei é svolto dai \textit{codici binari minimali} - codici istantanei e completi per i primi $k$ numeri naturali che utilizzano un numero variabile di bit. Esistono diverse possibilitá per le scelte delle parole del codice\footnote{In realtá, un codice binario minimale é semplicemente un codice ottimo per la distribuzione uniforme, il che spiega perché sono possibili scelte diverse per le parole del codice}, ma in quanto segue diremo che il codice binario minimale di $n$ (nei primi $n$ naturali) é definito come segue: sia $s = \ceil{\log(k)}$; se $n < 2^s -k$ é codificato dall'$n$-esima parola binaria (in ordine lessicografico) di lunghezza $s - 1$; altrimenti, $n$ é codificato utilizzando la $(n - k + 2^s)$-esima parola binaria di lunghezza $s$.\\
%%%%%%%%%%%%%%%%%%%%%%%%%%%%%%%%%%%%%%%%%%%%%%%%%%%%%%%%%%%%%%%%%%%%%%%%%%%%%%%%%%%%%

%                    TABELLA BINARIO MINIMALE

%%%%%%%%%%%%%%%%%%%%%%%%%%%%%%%%%%%%%%%%%%%%%%%%%%%%%%%%%%%%%%%%%%%%%%%%%%%%%%%%%%%%%
\noindent La base di tutti i codici istantanei per gli interi é il codice \textit{unario}. Il codice unario rappresenta il naturale $n$ tramite $n$ uno seguiti da uno 0\footnote{Nelle note originali il professore definisce l'unario all'incontrario e mette in una nota quello che ho definito io, ma é evidente che questa definizione é piú importante all'atto pratico per il semplice fatto che fa coincidere ordine lessicografico e l'ordine dei valori rappresentati}. Le prime parole del codice sono 0, 10, 110, \dots. La lunghezza di una parola in unario é banalmente $n + 1$. Il codice é sia istanteneo e completo.\\
Il codice $\gamma$ % INSERIRE LA CITAZIONE %
codifica un intero $n$ scrivendo il numero di bit della rappresentazione ridotta in unario, seguito dalla rappresentazione binaria ridotta di $n$. Le prime parole del codice sono 1, 010, 011, 00100, 00101, \dots. La lunghezza della parola di codice é quindi $2\lambda(n + 1) + 1$ e il codice é sia istantaneo che completo, questo si deve al fatto che l'unario é istantaneo e completo.\\
Analogamente, il codice $\delta$ % INSERIRE LA CITAZIONE %
codifica un intero $n$ scrivendo il numero di bit della rappresentazione binaria ridotta di $n$ in $\gamma$, seguito dalla rappresentazione binaria ridotta di $n$. Le prime parole del codice sono 1, 0100, 0101, 01100, 01101, \dots. La lunghezza della parola di codice per $n$ é quindi $2\lambda(\lambda(n + 1) + 1) + 1 + \lambda(n + 1)$ e il fatto che il codice sia istantaneo e completo deriva dal fatto che il $\gamma$ lo sia.\\
Si potrebbe provare a continuare in questa direzione, ma come vedremo, senza vantaggi significativi.\\
Il \textit{Codice di Golomb di modulo $k$} % INSERIRE LA CITAZIONE %
codifica un numero intero $n$ scrivendo il quoziente della divsione di $n$ per $k$ in uniario, seguito dal resto in binario minimale. Le prime parole del codice per $k = 3$ sono 10, 110, 111, 010, \dots. La lunghezza della parola di codice per $n$ é quindi $\flr{n/k} + 1 + \lambda(x\mod(k)) + [x\mod(k) \geq 2^{\ceil{\log(k)}} - k]$ e il fatto che sia istantaneo e completo deriva dal fatto che lo sono sia il codice unario che il codice binario minimale.\\
Infine conviene ricordare i \textit{codici a blocchi di lunghezza variabile}, come il codice variabile a nibble o a byte. L'idea é che ogni parola é formata da un numero variabile di blocchi di $k$ bit (4 nel caso dei nibble e 8 nel caso dei byte). Il primo bit non é codificante ed é noto come \textit{bit di continuazione}, se posto a 1 il blocco che stiamo considerando non é quello finale, se posto a 0 abbiamo raggiunto il blocco terminale.\\
In fase di codifica un intero $n$ viene scritto in notazione binaria, allineato a un multiplo di $k - 1$ bit, diviso in blocchi di $k - 1$ bit, e rappresentato tramite una sequenza di suddetti blocchi, ciascuno preceduto dal bit dicontinuazione. La lunghezza della parola di codice é pari a $\ceil{(\log(x) + 1) / k}(k + 1)$. I codici a lunghezza variabile sono ovviamente istantanei ma non completi, questo perché sequenze di 0 che sono piú lunghe di un blocco non sono confrontabili con nessuna delle parole del codice. Uno standard alternativo per questo tipo di codici é quello implementato da UTF-8, che anziché perdere il primo bit di ogni blocco per i bit di continuazione, sfrutta il primo blocco dell'intera parola per codificare quanti saranno i blocchi costituenti la parola.
%%%%%%%%%%%%%%%%%%%%%%%%%%%%%%%%%%%%%%%%%%%%%%%%%%%%%%%%%%%%%%%%%%%%%%%%%%%%%%%%%%%%%

%                    TABELLA RIASSUNTIVA

%%%%%%%%%%%%%%%%%%%%%%%%%%%%%%%%%%%%%%%%%%%%%%%%%%%%%%%%%%%%%%%%%%%%%%%%%%%%%%%%%%%%%
\subsection{Caratteristiche matematiche dei codici}
Esistono delle caratteristiche intrinseche dei codici istantanei per gli interi che permettono di classificarli e distinguerne il comportamento. In particolare, un codice é \textit{universale} se per qualunque distribuzione $p$ sugli interi monotona non crescente ($p(i) \leq p(i + 1)$)il valore atteso della lunghezza di una parola rispetto a $p$ é minore o uguale dell'entropia di $p$ a meno di costanti additive e moltiplicative indipendenti da $p$. Ció significa che se $l(n)$ é la lunghezza dlela paola di codice per $n$ e $H(p)$ é l'entropia di una distribuzione $p$ (nel senso di Shannon), esistono $c, d$ costanti tali che:
\begin{equation*}
    \sum_{x \in \nat}{l(n)p(n)} \leq cH(p) + d
\end{equation*}
Un codice é detto \textit{asintoticamente ottimo} quando a destra il limite superiore é della forma $f(H(p))$ con $\lim_{n \to \infty} f(n) = 1$.\\
Il codice unario e i codici di Golomb non sono universali, mentre lo sono $\gamma$ e $\delta$ inoltre, quest'ultimo, é anche asintoticamente ottimo.