\section{Una nozione astratta di sistema per il reperimento di informazioni}
Nella sua accezione più generale, un sistema di reperimento di informazioni (\textit{information retrieval system}) è dato da una collezione documentale \set{D} (insieme di documenti) di dimensione \set{N}, da un insieme \set{Q} di interrogazioni, e da una funzione di ranking \fun{r}{Q \times D}{\reals} che assegna a ogni coppia data da un'interrogazione e un documento, un punteggio (un numero reale), L'idea è che a fronte di un'interrogazione, a ogni documento viene assegnato un punteggio: i documenti con punteggio nullo non sono considerati rilevanti, mentre quelli a punteggio non nullo sono tanto più rilevanti quanto più il punteggio è alto.\\
Fissata un'interrogazione $q$ il sistema assegna un rango (cioè una graduatoria) tra i documenti rilevanti, ordinandoli per punteggio; il rango è essenziale per restituire i documenti in un ordine specifico, in particolare quando la collezione documentale è di grandi dimensioni.\\
I criteri per assegnare punteggi si dividono in \textit{endogeni} ed \textit{esogeni}. I due termini (non completamente formali) distinguono punteggi che utilizzano il contenuto del documento (cioè l'interno) da quelli che utilizzano la struttura esterna(per esempio, il grafo dei collegamenti ipertestuali tra i documenti). I criteri si dividono ulteriormente in \textit{statici} (o indipendenti dall'interrogazione) e \textit{dinamici} (o dipendenti dall'interrogazione). Nel primo caso, il punteggio assegnato a ciascun documento è fisso. Tutte le misure di centralità che abbiamo discusso possono essere utilizzate come punteggi esogeni, e quindi ci concentreremo su quelli endogeni.\\
La valutazione di un information retrieval system è basata sull'assunzione che a ogni interrogazione $q$ sia assegnato un insieme di documenti \textit{rilevanti} - quelle che un ipotetico utente considererebbe risposte valide all'interrogazione stessa. Il concetto può essere raffinato assumendo un ordine (totale o parziale) sui documenti rilevanti, che deve essere il più possibile coincidente con il punteggio assegnato dal sistema.