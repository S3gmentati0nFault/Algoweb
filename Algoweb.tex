\documentclass[a4paper]{book-enhanced}
\usepackage{src/preambolo}

\title{Nozioni per il corso di Algoritmica per il Web}
\subtitle{Dispense originali a cura di Sebastiano Vigna}
\author{Revisione di Alessandro Biagiotti}
\university{Università degli Studi di Milano}
\dept{Dipartimento di Informatica}
\logo{img/logo.tikz}
\subver{1.2}

\begin{document}

\maketitle
\noindent Le dispense di seguito presentate sono ad opera del professor Sebastiano Vigna (la
revisione è ad opera di Alessandro Biagiotti), si tratta di un collage di tutta una serie di
informazioni ricavate nell'arco del processo di preparazione all'esame e post superamento, nel tentativo di fornire a chiunque voglia preparare Algoritmica per il Web il miglior materiale possibile per farlo.

Vorrei prendere un minuto per ringraziare tutte le persone che hanno aiutato nel processo:
\begin{itemize}
    \item\textbf{Alessandro Clerici} - fonte di ispirazione, ha apportato grosse modifiche a una delle versioni precedenti (1.1.3) fornendo inoltre il codice della sua classe personalizzata, che ha aiutato a migliorare notevolmente il materiale. 

    \item\textbf{Chiara Prezioso} - per l'aiuto importante che ha deto nel formalizzare e correggere tutte le parti di algebra. 

    \item\textbf{Davide Polidori} - il quale mi ha prestato i suoi appunti (che ancora dimorano a casa mia), senza i quali alcuni passaggi sarebbero tuttora oscuri.

    \item\textbf{Matteo Silla} - Per aver fornito importanti correzioni concettuali a varie parti delle dispense.
\end{itemize}

Vorrei chiarire che le dispense sono una semplice riscrittura e ampliamento del materiale originale, non si tratta di materiale interamente originale, come dicono gli inglesi "\textit{if something ain't broken, don't try to fix it}".

Se trovate il materiale utile fatemelo sapere lasciando una stellina su \href{https://github.com/S3gmentati0nFault/Algoweb}{github}, sono sempre apprezzate, e fatelo girare anche tra i colleghi che sono in difficoltà con l'esame.

\tableofcontents
\clearpage

\incl{NozioniDiBase}
\incl{Crawling}
\incl{Tecniche_di_distribuzione_del_carico}
\incl{Codici_istantanei}
\incl{Gestione_della_lista_dei_termini}
\incl{Risoluzione_delle_interrogazioni}
\incl{Centrality}
\incl{Information_retrieval}
\incl{Punteggi_endogeni}
\incl{Richiami_di_algebra_lineare}
\bibliography{Bibliography}
\bibliographystyle{alpha}
\incl{License}

\end{document}
